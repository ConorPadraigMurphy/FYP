\chapter{Conclusion}
This dissertation has explored the development and deployment of TrafficVision, a web application that aims to transform urban mobility through real time traffic monitoring and advanced analytics. The main focus of this large scale project was to improve the network reliability of public transport services, commuter wait times and sustainable urbanization through traffic congestion and emissions reduction.

TrafficVision was developed using OpenCV for video processing, YOLOv8 for object detection, Kafka for data handling and MongoDB for data storage. These technologies were combined to process real-time video data. Database management systems was crucial for analyzing urban traffic patterns and improving public transport reliability and punctuality.

On a closer look, TrafficVision was a success in achieving its initial goals due to its real-time traffic status reports and bus schedule compliance. This critical feature allows commuters and urban planners to make informed decisions that significantly improved commuting efficiency and traffic control. However, TrafficVision currently has a limit on the amount of crowdsourced data it processes. With more data collection the system will become more precise and useful in terms of deep insights and more accurate traffic forecasts.

One of TrafficVision's key functions is providing transportation authorities and government entities with real-time information about bus punctuality and traffic density. This info is clear evidence that could be used to lobby for required reforms in the public transport system. TrafficVision identifies bus service problem areas and peak times with high delays and supports targeted interventions to reduce traffic congestion and improve bus service efficiency.

\subsection{Future improvements}
Future improvements to TrafficVision are numerous. Future developments could mainly focus on extending data collection techniques to cover more passive data streams and on enhancing system robustness to various environmental conditions. High-end image processing technologies could be combined to work well in different lighting and weather conditions to achieve uniform performance regardless of external conditions.

Furthermore, collaboration with local governments and transportation authorities could enhance data quality and comprehensiveness and extend TrafficVision's reach across geographic and operational boundaries. Such collaborations could also enable TrafficVision to be incorporated into official traffic management and urban planning initiatives as a core technology for smart city initiatives.

TrafficVision provided a solid foundation for the future of traffic management systems and demonstrated the potential of technology in resolving challenging urban mobility issues. The user-contributed data will continue to refine TrafficVision as a useful tool for urban planning and public transportation optimization. With ongoing technological developments and enhanced user engagement, TrafficVision will be considered a key element of future smart cities and will lead to transformation of public transportation.

\subsection{In conclusion}
TrafficVision met its initial objectives of proving it was capable of boosting the public transportation reliability via real time traffic monitoring. However with increased time for deployment and a bigger dataset the application would have further refined during its first functional stage. These enhancements would refine the system's accuracy and user interface, setting the foundation for its subsequent development stages.

Looking forward, machine learning could be used to analyze the massive traffic data collected. This advancement will enable TrafficVision to not just react to actual traffic situations but also predict the real traffic patterns to better inform urbanized transportation methods. Such predictive features, coupled with environmental traffic monitoring (like emissions monitoring),would expand TrafficVision into a more of a tool for city planning and sustainability.