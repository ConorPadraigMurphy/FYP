\chapter{Introduction}
This dissertation presents TrafficVision, a new web application developed to show bus system reliability and punctuality for public transport. Consistent bus schedules result in issues for commuters, the infrastructure and the environment of cities in urbanized areas where efficient transportation is a core element of everyday life. TrafficVision is built for these challenges with computer vision and web technologies, providing real-time information about bus service schedules and quality.

\section{Context and Relevance}

Urban mobility relies on public transportation, particularly bus systems. Their effective operation impacts the environmental sustainability and the daily life of the population. However, bus services often have delays and unpredictability that inconvenience passengers and lead to higher carbon emissions and congestion. Such inefficiencies hinder sustainable urban living. TrafficVision hopes to ease these issues by offering real time bus info so commuters, transportation authorities and urban planners can make the best choice possible.

\section{Project Objectives}

The primary objectives of TrafficVision are:

\begin{itemize}
    \item To develop a web-based platform that offers real-time tracking and analysis of bus services.
    \item To enhance the punctuality and reliability of public transportation through data-driven insights.
    \item To improve the daily commuting experience by minimizing uncertainty and wait times for bus passengers.
    \item To assist in reducing traffic congestion and carbon footprint through efficient bus service management.
\end{itemize}

These objectives form the benchmarks against which the project's success will be evaluated, with specific metrics detailed in the subsequent chapters.

\section{Dissertation Overview}

This dissertation reports on the development of TrafficVision from concept to implementation, highlighting its challenges and creative solutions. It is structured into several key sections:

\begin{itemize}
    \item \textbf{Chapter 1: Introduction} This chapter introduces TrafficVision, a web application aiming to improve public transport reliability by providing real-time bus service data. It outlines the urban mobility challenges and the project’s objectives to enhance bus service punctuality, improve commuting experiences and support sustainable urban living.
    
    \item \textbf{Chapter 2: Methodology} The methodology chapter details the structured approach to software development adopted for the TrafficVision project, including the use of Kanban boards and Gantt charts for project management, Agile methodologies for iterative development and the use of GitHub for collaboration and version control.
    
    \item \textbf{Chapter 3: Technology Review} In this chapter, the technology choices for TrafficVision are justified, including the use of MongoDB for its flexible data schema and cloud integration, Python for computer vision tasks, YOLOv8 over TensorFlow for efficient real-time object detection, React for UI development and Node.js for server-side scripting.

    \item \textbf{Chapter 4: System Design} This chapter covers the system architecture and design, explaining the use of React for front-end deployment, serverless architecture for the processing server, Kafka for data flow management, MongoDB for data storage and the integration of these technologies to create a scalable and responsive system.

    \item \textbf{Chapter 5: System Evaluation} The system evaluation chapter assesses TrafficVision against its initial objectives, presenting results and discussing the system's strengths and limitations. It examines real-time traffic monitoring capabilities, the reliability of public transportation data provided, user engagement and interaction with the system and areas for future development.

    \item \textbf{Chapter 6: Conclusion} The conclusion revisits the project's context and objectives, summarizing key findings from the system evaluation, the challenges faced and the innovative solutions implemented. It encapsulates the project’s achievements and the potential for future work to enhance the TrafficVision system.

\end{itemize}

\section{Project Resources}

The project's source code and resources are available on GitHub, which served as the code repository throughout the development process. The repository URL is:

\begin{center}
\url{https://github.com/ConorPadraigMurphy/FYP}
\end{center}

The repository contains the following main elements:

\begin{itemize}
    \item \textbf{Codebase} - The complete source code for the TrafficVision application, including frontend, backend and computer vision components.
    \item \textbf{Documentation} - A comprehensive set of documentation detailing the system's setup, deployment and usage instructions.
    \item \textbf{Issue Tracker} - A record of issues, enhancements and tasks managed throughout the project lifecycle.
\end{itemize}